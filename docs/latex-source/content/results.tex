% !TEX root =  ../report.tex
\section{Results}

The team of evaluators reported diverse problems some of which appeared more often in the different reports whereas some of which appeared only in a single evaluator's submission. Our team categorizes the problems of such with higher or lower impact based on the significance of the problem. The problems are grouped based on the heuristic they belong to. 

\subsection{Visibility of system status}
A serious problem reported in an evaluation is that the 'change server connection' button in the home page uses an icon that is misleading. The icon that we are currently using is a cogwheel one whereas there are more suitable options. Furthermore,  a severe problem that has been spotted by an evaluator is the lack of indication whether the connection is stable. Moreover, an reviewer reported that the user is not able to edit a sub-task of a card once it has been added - only deletion is possible. Last but not least, an evaluator reported that the user cannot distinguish whether the board they have joined is read-only or read and write which is an issue of high importance because it prevents the user from having all of the necessary information to use the system. All of the aforementioned problems besides not showing status regarding the connection, were not frequently reported but our team is of the opinion that all of them are of high importance and adjustments should be made based on every reported issue. 

\subsection{Match between the system and the real world}
Two of the evaluators reported that the color-picker tool is not optimal for the regular user. They recommended using hex codes instead of RGB values for the colors and also stated that "color scheme" is not a particularly user-friendly term. Another severe problem that was reported was the lack of explanation for the read-only and the write-only codes since some users may not know what those terms mean. Furthermore, a serious problem that was spotted by an evaluator was that the "add list" button is not intuitive enough. Finally, one evaluator recommended that we indicate that the user will also join the board after creating it on our board-creation screen.

\subsection{User control and freedom}
A severe problem reported in 4 out of the 5 evaluations is the lack of a way to reverse changes done by the user. Currently, there are also no warnings displayed when a delete button is pressed. Several reports mentioned that keyboard shortcuts for reversing these changes could be added. Another often reported problem is the password system. Server connection is mentioned in multiple reports, highlighting the lack of a clear disconnect button for the server. In 2 reports it is pointed out that there is no way to set, change, or remove passwords from a board, which we consider to be a serious problem. The evaluators have reported multiple issues with the board view frame. The most severe of these and also the most frequently reported issue is the lack of a delete button for lists. 

\subsection{Consistency and Standards}
The problems that were reported the most are related to the size of some pop-ups and the placement of some components. The third and fourth provided mock-ups will have the same dimensions in the final product, this was an inconsistency in the provided designs. It was also pointed out that we do not have a button in the board settings referencing to the color wheel. In the final product when clicking the square with the color a pop-up containing a color picker will be opened.It was also mentioned that we should change the "X" with something more intuitive but we believe that this is more accurate to represent this button's functionality.

\subsection{Error prevention}

Most of the problems for this heuristic relate to the absence of data validation. For the server address (cf. Figure \ref{fig:server-frame}), it is possible that it does not correspond to any URL, and the user can be unfamiliar with how the URL is structured. Errors are therefore likely to happen here. Likewise for the board code (Figure \ref{fig:board-join}), since it consists of a long hexadecimal string, typing it will surely lead to errors and the only viable alternative is copying and pasting. Another possible error that could be prevented is accidental deletion of lists and cards. If that happens and there is no confirmation dialogue (or alike), the user could permanently and non-reversibly delete a component which he did not want to. Additionally, validation needs to be done for empty fields of a card (Figure \ref{fig:card-popup}), and in a similar manner for tags (Figure \ref{fig:tag-settings}), lists and boards (Figure \ref{fig:board-create}).

\subsection{Recognition rather then recall}
The most frequently reported issue was that the past server connections are not cached and it would have been easier if the user is presented by either a default or the most recent server connection address. Furthermore, a common reported problem was that the user is not able to see whether the cards have descriptions and if they do - what are those descriptions. Finally, a evaluator reported that \ref{fig:card-popup}, \ref{fig:tag-settings} are too detailed and appear too complicated for the user. We, the team members, believe that this makes the system really inconvenient to use and we should definitely make adjustments in order to resolve this issue. In conclusion, the reported problems connected to this heuristic are not a large number but changing our system in order to fix them should be prioritized. 

\subsection{Flexibility and efficiency of use}
The most severe problem, reported in all of the five evaluations, was by far the lack of keyboard shortcuts and/or hotkeys. Most of those evaluations mentioned that we should have keyboard shortcuts to access a help page by pressing the "?" button, we should enable opening menus or activating create or join actions by pressing the enter key. It was also reported that we should implement a page that explains each shortcut that is available in the application. Another frequently reported problem was the difficulty and inefficiency of editing boards, lists, and cards. Firstly, the user should be able to input all of the information of the card while creating it, rather than having to create it using only a title and then having to edit the rest of the information. Furthermore, most of the evaluators recommend having a double-click to edit a title feature for the boards, the lists, and the cards. Finally, one evaluator recommended double-clicking tags to open their edit pop-up. An evaluator noticed the severe problem which is the lack of an admin control page. It was recommended to add a home page that allows the user to choose either the user or the admin view.

\subsection{Aesthetic and minimalist design}
Most common problem mentioned is the lack of colors in our design. The most severe is the lack of color for tags, but the overall lack of any color design for the windows should also be considered. One evaluator pointed out that the card should not display the names of the tags because the card should only display an overview of the information. Displaying the names of all tags would be repetitive. In one report, the evaluator pointed out that the information displayed in the sub-task progress bar and the ratio of completed sub-tasks displayed the same information. In addition, the sub-task completion bar takes up too much space and does not give a lot of insight in the progress when there are many sub-tasks to complete.


\subsection{Help users recognize, diagnose, and recover from errors} \label{9-help}
The problem that occurred in all of the evaluations was to lack of a "Help" page that should guide the users through the shortcuts and the different functionalities of the buttons. In order to fix this it was agreed in the team that we will have a button that will open the help page. Another topic that was mentioned was the absence of the error messages when the user did something wrong, this will be added in the final product and will appear whenever the user does something that he is not supposed to.

\subsection{Help and documentation}

The most severe issue is the absence of a help page, as covered in section \ref{9-help}. Moreover, when errors happen, they should be displayed in a helpful way to help troubleshooting. 