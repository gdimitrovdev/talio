% !TEX root =  ../report.tex
\section{Conclusions and Improvements}

%1
After taking into considerations all reviews about the visibility of the system status we came to the conclusions that the button's icon for changing the server on the homepage should be changed. Another icon or a different type of indication should be inserted that should inform whether the server connection is stable. We should add a functionality to the sub-tasks of the cards so they can be editable. Finally, we can add an icon or a label indication whether the board is read-only or read-write.

%2
After considering the problems described by the evaluators regarding the match between the real world and the system, we came to the conclusion that we should use hex values instead of RGB ones in the color-picking tool. We couldn't find a more suitable term than "color scheme", so we will not be changing that. We are also going to be adding help buttons next to the read-only and write-only codes, explaining what each of these codes does. Another change that we are going to implement is recoloring the "add list" button in the board overview, as well as showing a tool tip when the user hovers over it in order to make it more intuitive. We will also change the label on screen number 4 to "Create and join a new board".

%3
To allow users to reverse deletions or other changes, a keyboard shortcut should be added. Before deleting lists and other objects containing a lot of information, a warning pop-up should be added, asking user whether they want to delete the object. The password system should be added, allowing the user to change, add and remove passwords for the boards. A delete button for lists should be added and a visible server disconnect button should be added

%4
After reviewing the evaluations received we have decided that figure \ref{fig:board-join} and figure \ref{fig:board-create}  will have the same sizes.  We will keep the "X" icon in figure \ref{fig:home-frame} because when clicking it the board will not be deleted from the system, it will simply be removed from the respective user's homepage and will need to be re-added later. Also we have decide to make it possible to access the color wheel for both the board and list costumization.

%5
For error prevention, the application should ask for the Server IP and port separately, because leaving the URL for the user is not appropriate for the general user. This will also avoid errors with the formatting of the data. The board code could in theory be changed to a name (or short code) with a password. This will make it easier to memorise the board information and more intuitive for the general user. The accidental deletion of a component can be avoided (or at least minimised) by always having confirmation dialogues for this sort of operation. The validation of data fields can be done and information popups can be shown if the data is invalid.

%6
TO make the system more convenient to use we came to the conclusion that we should set a by default value for the server address in the 'connect to a server scene' and we will also change the structure of the pop-up for the card and tag details so they do not have such a complicated layout. Finally, we will add an icon that indicates that when a card has a description.

%7
Regarding the flexibility and efficiency of use, we will be adding all of the keyboard shortcuts that are mentioned in the backlog. We will be implementing help screens, as well as a screen containing information regarding the hotkeys. We will change the card creation form to include more fields (such as description). Double-clicking for editing titles is already implemented for some components, so we are going to implement it for the ones for which it isn't. We are going to have an option for the user to open an admin panel.

%8
To make the design more visually pleasing, colors should be added to tags and to the design overall. Repetitive information such as tag names and the sub-task completion progress bar or ratio should be removed or reformatted.

%9
Regarding the absence of a "help" page we have decided to add it once probably in figure \ref{fig:home-frame}. This page will guide the user through the different button functionalities and and keyboard shortcuts.  We will also add error messages when the user does something not allowed.
 